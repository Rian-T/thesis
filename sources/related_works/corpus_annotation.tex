%%%%%%%%%%%%%%%%%%%%%%%%%%%%%%%%%%%%%%%%%%%%%%%%%%%%%%%%%%%%%%%%%%%%%%%%
\chapter{Pretraining Data: Corpora and Knowledge}
\label{chap:rw-corpus}
%%%%%%%%%%%%%%%%%%%%%%%%%%%%%%%%%%%%%%%%%%%%%%%%%%%%%%%%%%%%%%%%%%%%%%%%

% TODO rediger proprement
% TODO: le titre du chapitre est provisoire, a discuter


\section{Introduction}
\label{sec:corpus-intro}

Le chapitre precedent couvrait les modeles. Maintenant faut les nourrir.

Les LMs modernes sont data-hungry:
\begin{itemize}
  \item GPT-3 (Brown et al., 2020): 300B tokens
  \item LLaMA (Touvron et al., 2023): 1-2T tokens
  \item Chinchilla (Hoffmann et al., 2022): compute-optimal = plus de data, modele plus petit
\end{itemize}

La qualite et la composition du corpus de pretraining determinent largement les capacites du modele. "Data is the new bottleneck."

Deux types de donnees de pretraining:
\begin{itemize}
  \item Texte brut: web crawls, articles scientifiques, livres
  \item Connaissances structurees: ontologies medicales, knowledge graphs
\end{itemize}

Pourquoi c'est central pour la these:
\begin{itemize}
  \item Part 1 = construire un corpus biomed francais a partir de sources publiques
  \item Ch.6 = injecter des connaissances d'ontologies medicales pendant le pretraining
  \item Triple rarete: francais + biomedical + clinique
\end{itemize}

-> plan du chapitre: on part des corpus web generalistes, on zoome sur le filtrage qualite, puis sur les sources biomed specifiques, et enfin sur les ontologies medicales et comment les integrer dans le pretraining

% TODO rediger l'annonce du plan


\section{Web-Scale Pretraining Corpora}
\label{sec:corpus-web}

\subsection{Common Crawl}

Source de base de presque tous les grands corpus de pretraining:
\begin{itemize}
  \item Web crawl ouvert, ~2.5B pages par snapshot mensuel, ~250-400 TiB non compresse
  \item Snapshots mensuels depuis 2008, petaoctets cumules
  \item Contenu tres heterogene: articles, forums, spam, code, pubs, boilerplate...
  \item Pas utilisable directement -> necessite des pipelines de nettoyage lourdes
\end{itemize}

\subsection{Premiers grands corpus}

C4 (Raffel et al., 2020):
\begin{itemize}
  \item Colossal Clean Crawled Corpus, ~750GB d'anglais extrait d'un seul snapshot CC (avril 2019)
  \item Filtres: langdetect >= 99\% anglais, suppression phrases incompletes, dedup
  \item Dodge et al. (2021) documentent le contenu: brevets, sites militaires US, texte machine-generated, exemples de benchmarks
  \item Blocklist filtering retire disproportionnellement du texte de/sur les minorites
\end{itemize}

The Pile (Gao et al., 2020):
\begin{itemize}
  \item 800GB anglais, 22 sources diversifiees (PubMed, ArXiv, GitHub, StackExchange, Wikipedia, livres, etc.)
  \item Idee: diversite des sources > volume brut d'une seule source
  \item Open-source (EleutherAI)
\end{itemize}

OSCAR (Ortiz Suarez et al., 2019):
\begin{itemize}
  \item Pipeline asynchrone pour classifier CC par langue
  \item Optimise pour infra moyenne/basse ressource (I/O-bound)
  \item Produit le corpus de CamemBERT: 138GB de texte francais
  \item Shuffled au niveau ligne pour eviter les problemes de copyright (ironiquement)
\end{itemize}

\subsection{Corpus modernes (2024-2025)}

FineWeb (Penedo et al., 2024):
\begin{itemize}
  \item 15T tokens, 96 snapshots CC
  \item Documentation detaillee des strategies de dedup et filtrage
  \item Pipeline: extraction texte, filtrage langue, heuristiques qualite, dedup URL + MinHash
  \item FineWeb-Edu: sous-ensemble 1.3T tokens filtre pour contenu educatif (cf section suivante)
\end{itemize}

RedPajama-V2 (Weber et al., 2024):
\begin{itemize}
  \item Dataset ouvert pour entrainer des LLMs
  \item Inclut subset francais (RPv2-Fr)
  \item Metriques de qualite pre-calculees pour faciliter le filtrage
\end{itemize}

TxT360 (Tang et al., 2024):
\begin{itemize}
  \item "Perfect blend" de sources
  \item Dedup globale sur 99 snapshots CC: 20TB -> 4.83T tokens (reduction 80\%)
  \item Pipeline similaire a FineWeb + near-dedup globale additionnelle
\end{itemize}

Autres:
\begin{itemize}
  \item Dolma (Soldaini et al., 2024): 3T tokens, corpus ouvert pour recherche (OLMo)
  \item ROOTS (Laurencon et al., 2023): 1.6TB, multilingue composite (BigScience/BLOOM)
\end{itemize}

% TODO tableau comparatif des grands corpus

-> on a des milliards de tokens de texte web, mais la qualite varie enormement -> comment selectionner les bons documents?


\section{Data Quality and Curation}
\label{sec:corpus-quality}

\subsection{Filtrage heuristique}

Approches classiques, devenues standard depuis Gopher (Rae et al., 2022):
\begin{itemize}
  \item Regles: longueur min/max du document, ratio mots/symboles, proportion de majuscules, lignes dupliquees
  \item Filtrage par perplexite: KenLM entraine sur Wikipedia, documents a perplexite trop haute = bruit
  \item Language ID: fastText pour filtrer par langue
  \item Ces heuristiques sont reprises par presque tous les pipelines modernes (FineWeb, C4, Dolma)
\end{itemize}

Limites: les heuristiques sont necessaires mais insuffisantes. Elles retirent le bruit evident mais ne distinguent pas la qualite du contenu.

\subsection{Classifieurs de qualite}

Paradigme recent: LLM-as-annotator -> distillation vers petit classifieur -> filtrage a grande echelle.

FineWeb-Edu (Penedo et al., 2024):
\begin{itemize}
  \item Llama-3-70B-Instruct annote 460K pages web sur "educational value" (echelle 0-5)
  \item Echelle additive: le LLM evalue chaque critere et construit le score pas a pas
  \item Prompt cible le niveau primaire/college pour eviter de favoriser les papiers trop techniques
  \item Distillation vers classifieur de regression (Snowflake-arctic-embed, F1 = 82\% en binaire avec seuil >= 3)
  \item Classification de 15T tokens: 6K heures H100
  \item Retire 92\% des donnees, bat quand meme le corpus complet sur MMLU, ARC, OpenBookQA
\end{itemize}

DCLM / DataComp-LM (Li et al., 2024):
\begin{itemize}
  \item Classifieur FastText entraine pour separer instructions synthetiques (Open Hermes 2.5) du web general
  \item Favorise les structures Q\&A resolues: question courte + reponse directe + raisonnement bref
  \item Utilise dans le mix de pretraining d'OLMo2
\end{itemize}

Nemotron-CC (NVIDIA, 2024):
\begin{itemize}
  \item Classifieur de qualite multi-criteres (accuracy, clarity, coherence, etc.)
  \item Approche plus granulaire que le score educatif unique de FineWeb-Edu
\end{itemize}

% TODO verifier les details de Nemotron-CC vs Nemotron-4 15B

\subsection{Organisation par domaine}

WebOrganizer (Wettig et al., 2025):
\begin{itemize}
  \item Organise le web en taxonomies par topic et par format
  \item Annotations LLM distillees vers classifieurs efficaces
  \item Montre que FineWeb-Edu a un biais implicite vers Science \& Technology et Academic Writing
  \item L'efficacite de FineWeb-Edu vient en partie de preferences de domaine alignees avec les benchmarks (MMLU, HellaSwag)
  \item Combiner organisation par domaine + filtrage qualite > qualite seule
\end{itemize}

\subsection{Deduplication}

\begin{itemize}
  \item URL-level: retirer les duplicats exacts d'URL
  \item Document-level: MinHash + LSH pour near-duplicates
  \item TxT360: dedup globale sur 99 snapshots CC, reduction de 80\%
  \item Crucial: la duplication biaise le training vers les textes les plus copies (boilerplate, templates)
\end{itemize}

\subsection{Contamination}

Le filtrage agressif pose un probleme de contamination des benchmarks:
\begin{itemize}
  \item Deng et al. (2024): ChatGPT et GPT-4 devinent 52\% et 57\% des options manquantes de MMLU -> les QA pairs sont sur le web
  \item Xu et al. (2024): survey des techniques de detection de contamination dans les LLMs
  \item InfiniGram (Liu et al., 2024): outil pour identifier des exact matches dans les training sets
  %\item Les classifieurs entraines sur instructions/QA favorisent naturellement le format des benchmarks
  %\item Probleme: est-ce que les gains de FineWeb-Edu/DCLM viennent de la qualite ou de la contamination?
\end{itemize}

Tension fondamentale: filtrer plus agressivement = meilleurs scores benchmarks, mais potentiellement via contamination implicite plutot que vraie qualite.

% TODO: mentionner les approches de decontamination?

-> ok le filtrage marche pour les corpus web generalistes, mais en biomedical c'est different: les corpus sont plus petits, le domaine est tres specifique, les criteres de qualite ne sont pas les memes -> quelles sources biomed existent?


\section{Biomedical and Clinical Corpora}
\label{sec:corpus-biomed}

\subsection{Grandes sources anglophones}

PubMed:
\begin{itemize}
  \item Base de donnees de la NLM (National Library of Medicine)
  \item ~39M citations et abstracts d'articles biomedicaux
  \item Texte structure, vocabulaire technique, qualite editoriale
  \item Abstracts seulement (pas le full-text)
  \item Utilise par BioBERT (Lee et al., 2019), PubMedBERT (Gu et al., 2022)
\end{itemize}

PMC Open Access:
\begin{itemize}
  \item ~4.5M articles full-text en acces libre
  \item 98\% anglais (Labrak et al., 2024)
  \item Contenu heterogene: articles de recherche, reviews, case reports, editoriaux, guidelines
  \item Utilise par BioMistral (3B tokens), Meditron (46B tokens), PMC-LLaMA (75B tokens)
  \item Probleme: granularite article. Un article de recherche peut contenir a la fois des cas cliniques pertinents et des sections methodologiques non pertinentes
\end{itemize}

MIMIC (Johnson et al., 2016/2023):
\begin{itemize}
  \item MIMIC-III, MIMIC-IV: notes cliniques du Beth Israel Deaconess Medical Center
  \item Acces restreint (data use agreement, formation ethique)
  \item Utilise par ClinicalBERT (Alsentzer et al., 2019), Clinical Longformer (Li et al., 2022), BioClinical ModernBERT (Sounack et al., 2025)
  \item Gold standard pour le texte clinique anglais, mais un seul hopital americain -> pas generalizable
\end{itemize}

\subsection{Curation article-level pour le biomed}

Meditron (Chen et al., 2023):
\begin{itemize}
  \item Score 0-1 par article base sur: MeSH tags, type de publication, reputation du journal, recence, nombre de citations
  \item Upsampling des articles a haut score, filtrage des bas scores
  \item 46B tokens au total, 128x A100, 332h pour le 70B
  \item Limite: granularite article. Un article peut melanger contenu pertinent et non-pertinent
\end{itemize}

BioClinical ModernBERT (Sounack et al., 2025):
\begin{itemize}
  \item 169B tokens: PubMed + PMC + 20 datasets cliniques (MIMIC-III, MIMIC-IV, CheXpert Plus, etc.)
  \item SOTA actuel sur les benchmarks biomed anglais
  \item Mais: la plupart des sources cliniques necessitent des data use agreements -> pas vraiment "public"
\end{itemize}

\subsection{Sources francophones}

Ressources publiques existantes:
\begin{itemize}
  \item ISTEX: base de 27M references scientifiques, permet d'extraire des documents francais bio/med
  \item CLEAR (Grabar \& Cardon, 2018): notices medicamenteuses, articles d'encyclopedie medicale en francais technique et simplifie
  \item E3C (Magnini et al., 2021): corpus multilingue de cas cliniques, notices, resumes de these medicale
  \item HAL / Halvest (Kulumba et al., 2024): articles scientifiques francais et anglais extraits du depot HAL (Hyper Articles en Ligne)
  \item Wikipedia biomedical: articles des portails medecine, pharmacie, biologie. Libre et multilingue
\end{itemize}

Modeles entraines from scratch sur corpus francais:
\begin{itemize}
  \item DrBERT (Labrak et al., 2023): corpus de 7GB, from scratch sur 128x V100
    \begin{itemize}
      \item Pretend que le continual PT ne marche pas pour le francais biomed
      \item Mais: evaluation potentiellement biaisee (cf Ch.4 de la these)
    \end{itemize}
  \item AliBERT (Berhe et al., 2023): ScienceDirect + theses Sudoc, tokenizer Unigram regularise, 48x A100
    \begin{itemize}
      \item Modele non disponible publiquement
    \end{itemize}
\end{itemize}

Tentatives de continual pretraining en francais biomed:
\begin{itemize}
  \item Copara et al. (2020): 31K articles seulement -> aucune amelioration sur CamemBERT-large
  \item Le Clercq de Lannoy et al. (2022): PubMed + Cochrane + ISTEX + Wikipedia = 136M mots -> +2 F1 sur EMEA, rien sur MEDLINE
  \item Dura et al. (2022): 21M documents cliniques APHP -> +3\% sur APMed (prive), mais donnees non partageables
\end{itemize}

Resultats décevants: DrBERT dit que continual PT marche pas, Copara et Le Clercq confirment. Mais le probleme est peut-etre le corpus (trop petit, pas assez cible), pas la methode.

Triple rarete: francais + biomedical + clinique. Les textes cliniques sont prives (CNIL), les modeles hospitaliers ne sont pas partageables entre etablissements. Les sources publiques francophones sont fragmentees et petites comparees a PubMed/PMC.

% TODO tableau comparatif des corpus biomed (EN vs FR, taille, acces)

-> le texte biomedical aide pour le pretraining, mais les LMs captent des patterns statistiques, pas la structure des ontologies. Pour le codage medical (CIM-10, CCAM) il faut des connaissances hierarchiques explicites -> peut-on utiliser directement les ontologies comme donnees de pretraining?


\section{Medical Ontologies and Knowledge Graphs}
\label{sec:corpus-ontologies}

\subsection{Principales ontologies medicales}

UMLS (Lindberg et al., 1993):
\begin{itemize}
  \item Unified Medical Language System, meta-ontologie biomedical de la NLM
  \item Integre ~189 vocabulaires sources (SNOMED-CT, MeSH, ICD, ATC, etc.)
  \item ~3.4M concepts, ~16.7M noms de concepts uniques
  \item Standard international pour la recherche en NLP biomed
  \item Contient: synonymes, relations hierarchiques, relations associatives, definitions
\end{itemize}

SNOMED-CT:
\begin{itemize}
  \item ~371K concepts cliniques structures
  \item Relations hierarchiques (is-a) + associatives (finding site, causative agent, etc.)
  \item Standard pour les systemes d'information clinique internationaux
  \item Couverture: diagnostics, procedures, anatomie, substances, organismes
\end{itemize}

CIM-10 / ICD-10 (OMS, 2019):
\begin{itemize}
  \item Classification internationale des maladies
  \item Hierarchie: chapitres -> blocs -> categories -> sous-categories
  \item Version francaise enrichie par l'ATIH avec: exclusions, inclusions, notes, libelles courts/longs
  \item Utilise pour le codage PMSI (Programme de Medicalisation des Systemes d'Information) en France
  \item ~19K codes dans la version ATIH
\end{itemize}

CCAM (ATIH):
\begin{itemize}
  \item Classification Commune des Actes Medicaux, ~8K codes
  \item Specifique a la France, structure hierarchique
  \item Relations: actes associes, incompatibilites, modificateurs
\end{itemize}

ATC (OMS):
\begin{itemize}
  \item Anatomical Therapeutic Chemical, ~6K codes au 5eme niveau
  \item 5 niveaux: anatomique -> therapeutique -> pharmacologique -> chimique -> substance
  \item Standard pour la classification des medicaments
\end{itemize}

\subsection{Representations formelles}

\begin{itemize}
  \item RDF (Resource Description Framework): triplets sujet-predicat-objet, standard W3C
  \item OWL (Web Ontology Language): logique descriptive, raisonnement automatique, classes, proprietes
  \item SKOS (Simple Knowledge Organization System): prefLabel, altLabel (synonymes), broader/narrower (hierarchie), related, exactMatch
\end{itemize}

Les ontologies codent explicitement ce que les LMs doivent apprendre implicitement:
\begin{itemize}
  \item Hierarchie: E11 (diabete type 2) $\subset$ E10-E14 (diabete sucre)
  \item Exclusions: E11 $\neq$ E10 (diabete type 1) -- a ne pas confondre
  \item Synonymes: un meme concept a plusieurs libelles (prefLabel + altLabel)
  \item Relations associatives: complications, traitements, liens entre ontologies
\end{itemize}

-> les ontologies sont riches en connaissances structurees, mais sous forme de graphes: pas directement exploitables par les LMs qui attendent du texte en language natu -> comment les integrer dans le pretraining?


\section{Knowledge-Enhanced Pretraining}
\label{sec:corpus-knowledge}

\subsection{Embeddings statiques de graphes}

Approches qui produisent des vecteurs fixes (pas contextuels) a partir de graphes:
\begin{itemize}
  \item Node2Vec (Grover \& Leskovec, 2016): random walks sur graphe -> skip-gram, apprentissage de representations de noeuds
  \item RDF2Vec (Ristoski \& Paulheim, 2016): adapte le paradigme walk + skip-gram aux graphes RDF. Fondation des approches de walks sur ontologies
  \item Snomed2Vec (Agarwal et al., 2019): Node2Vec applique a SNOMED-CT, 5-6x amelioration sur concept similarity
  \item OWL2Vec* (Chen et al., 2021): walks sur ontologies OWL (teste sur Gene Ontology)
\end{itemize}

Limite: embeddings statiques, pas integres dans un transformer. Un concept = un vecteur fixe quel que soit le contexte. Pas de contextualisation.

-> on peut faire mieux en integrant les KG directement dans les transformers

\subsection{Injection de connaissances dans les transformers}

Premiere generation: injecter des entites dans l'architecture:
\begin{itemize}
  \item ERNIE (Zhang et al., 2019): aligne entity embeddings (TransE) avec representations textuelles pendant le pretraining. Entity linking implicite
  \item KnowBERT (Peters et al., 2019): integre des entity linkers comme couches d'attention supplementaires dans BERT
  \item K-BERT (Liu et al., 2020): injecte des triples de KG comme soft-position embeddings dans l'input. Modifie la sequence d'entree directement
\end{itemize}

Deuxieme generation: multi-task pretraining language + KG:
\begin{itemize}
  \item KEPLER (Wang et al., 2021): MLM + TransE knowledge embedding sur Wikidata simultanement. Les deux objectifs partagent le meme encoder
  \item CoLAKE (Sun et al., COLING 2020): construit des word-knowledge graphs qui unifient contexte textuel et contexte KG, puis pretraine avec MLM modifie sur ces graphes
  \item DRAGON (Yasunaga et al., NeurIPS 2022): joint pretraining bidirectionnel MLM + link prediction sur UMLS
    \begin{itemize}
      \item PubMed pour le texte, UMLS pour le KG
      \item +3\% accuracy sur MedQA, +10\% sur raisonnement multi-step
      \item Reference pour le multi-task MLM + KG
    \end{itemize}
\end{itemize}

\subsection{Approches contrastives medicales}

\begin{itemize}
  \item SapBERT (Liu et al., NAACL 2021): self-alignment pretraining sur synonymes UMLS
    \begin{itemize}
      \item Contrastive learning: synonymes UMLS = positifs, non-synonymes = negatifs
      \item Metric learning scalable sur 4M+ concepts UMLS
      \item SOTA entity linking biomed sur 6 benchmarks
      \item Un seul modele pour tous les types d'entites ("one-model-for-all")
    \end{itemize}
  \item CODER (Yuan et al., 2022): contrastive pretraining sur 87M triplets de relations UMLS
    \begin{itemize}
      \item Relations + synonymes dans l'objectif contrastif
      \item Medical term normalization
    \end{itemize}
\end{itemize}

\subsection{Graph-to-text pour le pretraining}

Approche recente: transformer les graphes en texte lisible par les LMs.

"Textbooks Are All You Need" (Gunasekar et al., 2023):
\begin{itemize}
  \item Donnees synthetiques de qualite textbook generees par GPT -> petits modeles tres performants
  \item Montre que la qualite et la structure des donnees importent plus que le volume brut
  \item Paradigme: generer des donnees synthetiques structurees plutot que crawler du texte brut
\end{itemize}

AntGLM-Med (Zhou et al., 2023):
\begin{itemize}
  \item Reecrit des sous-graphes de KG medical en prose naturelle
  \item Utilise pour le continual pretraining de LLMs medicaux
  \item Plus proche precedent de la generation de texte a partir d'ontologies pour pretraining
\end{itemize}

% TODO verifier les details d'AntGLM-Med (Omaha KG?)

EntiGraph (Yang et al., ICLR 2025):
\begin{itemize}
  \item Algorithme d'augmentation entity-centric: decompose un corpus en entites puis genere du texte sur les relations entre entites
  \item 1.3M tokens reels -> 600M tokens synthetiques via GPT-4-turbo
  \item Scaling log-lineaire de l'accuracy avec le nombre de tokens synthetiques
  \item Cible les decoders (Llama 3 8B), pas encore teste sur encoders
\end{itemize}

Limites actuelles du knowledge-enhanced pretraining:
\begin{itemize}
  \item La plupart des travaux sont en anglais et sur UMLS
  \item Graph-to-text pour le pretraining est recent (2023-2025)
  \item Pas encore explore pour les encoders bidirectionnels (DRAGON fait du link prediction, pas du graph-to-text)
  \item Les ontologies nationales (CIM-10 ATIH, CCAM) n'ont jamais ete utilisees pour du pretraining
  \item SapBERT et CODER exploitent UMLS mais uniquement via contrastive learning, pas via generation de texte
\end{itemize}

-> methodes prometteuses mais gap important: langues non-anglaises, ontologies nationales, pretraining d'encoders via graph-to-text


\section{Limites et transition}
\label{sec:corpus-limitations}

\begin{itemize}
  \item Rarete biomed non-anglais: PMC = 98\% anglais, les corpus francais biomed sont petits et fragmentes. Les pipelines de filtrage (FineWeb-Edu, DCLM) sont concus pour l'anglais
    -> besoin de strategies de collection et de filtrage adaptees au domaine et a la langue
  \item Granularite de filtrage: les approches existantes (Meditron, FineWeb-Edu) filtrent au niveau document ou article. Mais un article biomed peut contenir a la fois des cas cliniques pertinents et des sections non pertinentes
    -> besoin de filtrage a granularite plus fine (paragraphe)
  \item Criteres de qualite non adaptes: les criteres "educatifs" de FineWeb-Edu ne correspondent pas au texte biomedical (cas cliniques, notices medicamenteuses, protocoles). Les biais vers Science \& Technology identifies par WebOrganizer peuvent etre differents en biomed
    -> besoin de criteres domaine-specifiques
  \item Ontologies nationales sous-exploitees: UMLS domine la recherche, mais les systemes hospitaliers francais utilisent CIM-10/ATIH, CCAM, ATC. Aucun travail n'a integre ces ontologies dans le pretraining
    -> gap pour le codage medical en francais
  \item Graph-to-text naissant: AntGLM-Med et EntiGraph montrent la voie mais sont limites aux decoders en anglais. Pas encore explore pour les encoders bidirectionnels ni pour les ontologies non-anglophones
    -> possibilite de generer du texte "textbook" a partir des ontologies ATIH
\end{itemize}

-> ce chapitre a couvert les donnees de pretraining: sources textuelles web et biomed, methodes de curation et filtrage, ontologies medicales et leurs integrations dans le pretraining. Le chapitre suivant traite de la tache finale: l'extraction d'information clinique, et de comment les modeles pretrained sont utilises pour cette tache.

% TODO rediger la transition proprement
