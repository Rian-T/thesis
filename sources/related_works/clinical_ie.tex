%%%%%%%%%%%%%%%%%%%%%%%%%%%%%%%%%%%%%%%%%%%%%%%%%%%%%%%%%%%%%%%%%%%%%%%%
\chapter{Clinical Information Extraction}
\label{chap:rw-ie}
%%%%%%%%%%%%%%%%%%%%%%%%%%%%%%%%%%%%%%%%%%%%%%%%%%%%%%%%%%%%%%%%%%%%%%%%

% TODO rediger proprement


\section{Introduction}
\label{sec:ie-intro}

Clinical Information Extraction (IE) = extraire de l'information structuree a partir de texte clinique non structure.

Pourquoi c'est important:
\begin{itemize}
  \item Les entrepots de donnees cliniques (CDW) se developpent dans les hopitaux
  \item Raghavan et al. (2014): jusqu'a 80\% des entites cliniques sont absentes des donnees structurees -> le texte libre (comptes-rendus, courriers, prescriptions) est la source principale
  \item En France: PMSI (Programme de Medicalisation des Systemes d'Information), chaque sejour hospitalier doit etre code avec des codes CIM-10 et CCAM. ~30M sejours/an (ATIH)
  \item Le codage est fait manuellement par des codeurs professionnels -> automatisation = enjeu majeur
\end{itemize}

Types de taches:
\begin{itemize}
  \item Named Entity Recognition (NER): identifier les mentions d'entites dans le texte
  \item Relation Extraction (RE): extraire les relations entre entites
  \item Medical coding: assigner des codes d'ontologie (CIM-10, CCAM, ATC) a un document
  \item Entity linking / normalization: associer une mention a un concept dans une base de connaissances (UMLS, SNOMED-CT)
\end{itemize}

Pourquoi c'est central pour la these:
\begin{itemize}
  \item Les taches de clinical IE sont le terrain d'evaluation final des modeles pretrained (Part 1-2)
  \item Les defis identifies ici (rarete des donnees, label space enorme, evaluation non standardisee) motivent les contributions de Part 3
\end{itemize}

-> plan du chapitre: on part des fondamentaux du NER, on zoome sur les benchmarks biomed, puis sur le codage medical, l'entity linking, les approches zero-shot, les donnees synthetiques, et enfin l'evaluation

% TODO rediger l'annonce du plan


\section{Named Entity Recognition}
\label{sec:ie-ner}

\subsection{Sequence Labeling}

NER = identifier et classifier les mentions d'entites dans un texte.

Formulation classique: sequence labeling avec schema BIO (Beginning, Inside, Outside):
\begin{itemize}
  \item Chaque token recoit un label: B-TYPE, I-TYPE, ou O
  \item Variantes: BILOU (Beginning, Inside, Last, Outside, Unit) pour plus de precision
\end{itemize}

CRF (Lafferty et al., 2001):
\begin{itemize}
  \item Conditional Random Fields: modele les dependances entre labels consecutifs
  \item Evite les sequences invalides (ex: I-MALADIE apres B-TRAITEMENT)
  \item Standard pre-deep learning pour le NER
\end{itemize}

BiLSTM-CRF (Lample et al., 2016):
\begin{itemize}
  \item Combine BiLSTM pour les representations contextuelles + CRF pour le decodage
  \item SOTA pre-BERT sur la plupart des benchmarks NER
  \item Character-level embeddings pour capter la morphologie
\end{itemize}

\subsection{BERT-based NER}

Devlin et al. (2019): BERT + linear layer sur chaque token -> NER.
\begin{itemize}
  \item Simple: pretrained encoder + classification head, fine-tuning end-to-end
  \item Pas besoin de features manuelles ni de CRF (debat: CRF aide parfois marginalement)
  \item Representations contextuelles profondes: ``cancer'' dans ``cancer du poumon'' vs ``cancer du sein'' -> vecteurs differents
  \item Transfer learning: le modele pretrained apporte des connaissances linguistiques generales
\end{itemize}

\subsection{Au-dela du token-level}

Span-based NER:
\begin{itemize}
  \item Au lieu de labelliser chaque token, on enumere les spans candidats et on classifie chaque span
  \item Avantage: gere mieux les entites imbriquees (nested entities)
  \item Exemple biomed: ``cancer du poumon droit'' = [MALADIE [ANATOMIE]]
  \item Les entites imbriquees sont frequentes en biomed: symptomes dans des maladies, anatomie dans des procedures
\end{itemize}

-> ok le NER marche bien en general, mais le domaine biomedical a ses specificites: vocabulaire technique, entites imbriquees, negation, style telegraphique


\section{Biomedical and Clinical NER}
\label{sec:ie-biomed-ner}

\subsection{Benchmarks anglais}

NER sur texte scientifique (PubMed):
\begin{itemize}
  \item BC5CDR (Li et al., 2016): chemical-disease relations, 1500 articles PubMed, 2 types d'entites (chemical, disease)
  \item NCBI Disease (Dogan et al., 2014): mentions de maladies dans 793 abstracts PubMed
  \item JNLPBA (Collier \& Kim, 2004): genes/proteines, corpus GENIA, 5 types d'entites
  \item AnatEM (Pyysalo \& Ananiadou, 2014): entites anatomiques, 1212 documents
  \item ChemProt (Krallinger et al., 2017): interactions chemical-protein, relation extraction
\end{itemize}

NER sur texte clinique:
\begin{itemize}
  \item i2b2 shared tasks (Uzuner et al., 2010, 2011): NER clinique sur notes MIMIC, plusieurs editions (medication extraction, coreference, relations temporelles)
  \item n2c2 (successeur de i2b2): n2c2 2018 (Henry et al., 2020) = adverse drug events + criteres d'eligibilite aux essais cliniques. n2c2 2022 = contextualization of clinical NLP
  \item Taches specifiques sur notes cliniques: de-identification (DEID), social determinants (SDoH), phenotyping
\end{itemize}

Classification:
\begin{itemize}
  \item GAD (Bravo et al., 2015): gene-disease associations, classification binaire
  \item Hallmarks of Cancer (Baker et al., 2016): classification multi-label de hallmarks tumoraux
\end{itemize}

BLURB (Gu et al., 2022): Biomedical Language Understanding and Reasoning Benchmark, unifie 13 datasets biomed en un seul benchmark. PubMedBERT = premier modele evalue.

\subsection{Benchmarks francais}

QUAERO (Neveol et al., 2014):
\begin{itemize}
  \item EMEA: notices medicamenteuses en francais
  \item MEDLINE: titres d'articles scientifiques
  \item 10 types d'entites suivant les semantic groups UMLS (Anatomy, Chemical, Device, Disorder, Gene, Living Being, Object, Phenomena, Physiology, Procedure)
  \item Entites imbriquees presentes
  \item Utilise dans CLEF eHealth 2016 (Neveol et al., 2016)
\end{itemize}

CAS / DEFT 2020 (Cardon et al., 2020):
\begin{itemize}
  \item Cas cliniques issus d'articles scientifiques francais
  \item Tache 3 de DEFT 2020 = extraction d'information clinique
  \item CAS1: 2 classes (pathologie, signes/symptomes)
  \item CAS2: 8 classes (anatomie, dose, examen, mode, moment, substance, traitement, valeur)
  \item Devenu benchmark standard pour le NER clinique francais
\end{itemize}

E3C (Magnini et al., 2021):
\begin{itemize}
  \item Corpus multilingue (EN, FR, ES, IT, EU)
  \item Cas cliniques, notices medicamenteuses, resumes de these
  \item Layer 1 = annote manuellement, layer 2 = semi-annote, layer 3 = non annote
  \item 1 seule classe: entite clinique (pas de sous-types)
\end{itemize}

CLEF eHealth 2016 (Neveol et al., 2016):
\begin{itemize}
  \item Task 2: NER + normalisation sur corpus QUAERO
  \item Codage CIM-10 sur certificats de deces (corpus CepiDC)
  \item BRATeval = outil d'evaluation officiel (exact match sur character offsets)
  \item 7 equipes participantes
\end{itemize}

DrBenchmark (Labrak et al., 2024, LREC-COLING):
\begin{itemize}
  \item Premier benchmark unifie pour le francais biomedical (equivalent de BLURB pour l'anglais)
  \item 20 taches sur 12 datasets: NER (9 taches), classification (8), QA (1), similarite semantique (2)
  \item Regroupe QUAERO, E3C, CAS/DEFT-2020, DEFT-2021, Mantra-GSC, FrenchMedMCQA, PxCorpus, MorFITT, CLISTER, DiaMed
  \item Permet enfin une comparaison equitable entre modeles sur le francais biomedical
  \item Resultat cle: les modeles specialises biomed surpassent les modeles generaux
\end{itemize}

\subsection{Specificites du texte clinique}

Le texte clinique est tres different du texte scientifique (PubMed) et du texte general (web):
\begin{itemize}
  \item Style telegraphique: phrases incompletes, listes de symptomes, abreviations non standard
  \item Abreviations: ``IDM'' (infarctus du myocarde), ``HTA'' (hypertension arterielle), ``NFS'' (numeration formule sanguine)
  \item Negation: ``pas de signe de pneumonie'' -> ``pneumonie'' est mentionnee mais absente. NegEx (Chapman et al., 2001): algorithme a base de regles pour detecter la negation, toujours utilise comme baseline
  \item Temporalite: antecedents vs diagnostics actuels vs hypotheses
  \item Incertitude: ``probable'', ``a surveiller'', ``a eliminer''
  \item Fautes d'orthographe frequentes (saisie rapide par les cliniciens)
\end{itemize}

Consequence: les modeles entraines sur du texte scientifique (PubMed) ou general (web) sont sous-optimaux pour le texte clinique -> d'ou les modeles pretrained sur le domaine (BioBERT, PubMedBERT, ClinicalBERT, GatorTron, DrBERT, cf \Cref{sec:lm-biomedical}). Ces modeles specialises ameliorent les performances sur les taches biomed, mais restent supervises: il faut annoter des exemples pour chaque classe cible.

Alternative architecturale: Clinical Mamba (Yang et al., 2024) adapte les State Space Models (SSM) au domaine clinique. Complexite lineaire en la longueur de sequence (vs quadratique pour les transformers) -> pertinent pour les documents cliniques longs. Premiers resultats competitifs avec les transformers.

% TODO tableau comparatif benchmarks EN vs FR

-> on a les benchmarks et les modeles pour le NER biomed, mais le codage medical c'est pas juste du NER: c'est assigner des codes d'une ontologie a un document entier, avec des milliers de labels possibles


\section{Medical Coding}
\label{sec:ie-coding}

\subsection{Le probleme}

Medical coding = assigner des codes standardises (CIM-10, CCAM, ATC) a un document clinique.

En France:
\begin{itemize}
  \item PMSI: chaque sejour hospitalier doit etre code
  \item DP (Diagnostic Principal), DAS (Diagnostics Associes), actes CCAM
  \item ~30M sejours/an (ATIH), code manuellement par des codeurs professionnels
  \item Erreurs de codage = impact financier et epidemiologique
\end{itemize}

Defis:
\begin{itemize}
  \item Label space enorme: ~17K codes CIM-10 (ATIH), ~8K codes CCAM, ~6K codes ATC
  \item Documents longs: comptes-rendus de sortie = souvent 1000-3000 tokens, depassent la limite de BERT (512)
  \item Distribution tres desequilibree: quelques codes frequents (I10 Hypertension, E11 Diabete), longue traine de codes rares
  \item Multi-label: un sejour peut avoir 1 DP + plusieurs DAS
\end{itemize}

\subsection{Approches}

CAML (Mullenbach et al., 2018, NAACL):
\begin{itemize}
  \item Convolutional Attention for Multi-Label classification
  \item CNN + label-wise attention: chaque code a son propre mecanisme d'attention sur le document
  \item Premiere approche explainable pour ICD coding: on peut visualiser quels passages du texte ont motive chaque code
  \item DR-CAML: regularisation par les descriptions textuelles des codes
\end{itemize}

HyperCore (Cao et al., 2020, ACL):
\begin{itemize}
  \item Exploite la hierarchie ICD via embeddings hyperboliques (Poincare ball)
  \item L'espace hyperbolique capture naturellement les structures arborescentes (distance exponentielle avec la profondeur)
  \item + graphe de co-occurrence entre codes (GCN)
\end{itemize}

PLM-ICD (Huang et al., 2022, ClinicalNLP@NAACL):
\begin{itemize}
  \item Premier a utiliser des pretrained language models pour ICD coding
  \item Label attention sur les representations du PLM
  \item Segment pooling pour gerer les documents longs (decoupe en segments, pool les representations)
  \item Resultat cle: meilleur encoder pretrained = meilleur codage -> la qualite du pretraining impacte directement le codage
\end{itemize}

MSMN (Yuan et al., 2022, ACL):
\begin{itemize}
  \item Multiple Synonyms Matching Network
  \item Exploite les synonymes UMLS pour chaque code ICD
  \item Le texte clinique utilise des expressions variees pour un meme concept: ``infarctus du myocarde'', ``IDM'', ``crise cardiaque'' -> tous I21
  \item Multi-synonym matching: representations de codes plus riches que la seule description officielle
\end{itemize}

Et les LLMs? Soroush et al. (2024, NEJM AI):
\begin{itemize}
  \item Evaluation de GPT-4 pour le codage ICD-10: exact match = 33.9\% seulement
  \item Les LLMs generatifs peinent sur le codage: label space trop grand, codes trop specifiques
  \item Les approches specialisees (PLM-ICD, MSMN) restent superieures
  \item -> renforce le besoin d'encoders specialises plutot que de LLMs generiques
\end{itemize}

\subsection{Codage en francais}

Ressources specifiques:
\begin{itemize}
  \item FRACCO (Pignat et al., 2024): corpus d'oncologie francais, 1301 cas cliniques annotes avec codes CIM-O-3.1, classification multi-label sur 100 codes
  \item FRASIMED (Zaghir et al., 2023): adaptation francaise de Cantemist (codage tumeurs, Miranda et al. 2020) et Distemist (codage maladies, Miranda et al. 2022), 2051 documents cliniques traduits de l'espagnol
  \item CepiDC / CLEF eHealth 2016: codage CIM-10 sur certificats de deces francais
\end{itemize}

% TODO tableau

-> le codage marche sur MIMIC (anglais), mais: 17K labels = impossible d'annoter des exemples pour chacun. En francais, les ressources sont encore plus rares. -> besoin d'approches qui generalisent a des labels non vus = zero-shot. Mais d'abord: comment associer mentions et concepts?


\section{Entity Linking and Normalization}
\label{sec:ie-linking}

\subsection{Le probleme}

Entity linking = associer une mention textuelle a une entite dans une base de connaissances (UMLS, SNOMED-CT, CIM-10).

Normalization = mapper des variantes terminologiques au concept canonique:
\begin{itemize}
  \item ``infarctus du myocarde'', ``IDM'', ``crise cardiaque'' -> I21.9 (CIM-10)
  \item ``diabete de type 2'', ``DNID'', ``diabete non insulinodependant'' -> E11 (CIM-10)
\end{itemize}

Difference avec le codage medical:
\begin{itemize}
  \item Entity linking = mention-level: associer chaque mention a un concept
  \item Medical coding = document-level: assigner des codes a un document entier
  \item Entity linking peut etre vu comme une brique du codage: d'abord extraire les mentions (NER), puis les normaliser (linking), puis agreger au niveau document
\end{itemize}

\subsection{Approches}

SapBERT (Liu et al., 2021, NAACL):
\begin{itemize}
  \item Self-Alignment Pretraining for BERT
  \item Contrastive learning sur synonymes UMLS: 4M+ concepts
  \item Synonymes = positifs, non-synonymes = negatifs
  \item SOTA entity linking biomed sur 6 benchmarks
  \item ``One-model-for-all'': un seul modele pour tous types d'entites (maladies, medicaments, procedures)
  \item Base: PubMedBERT fine-tune avec metric learning
\end{itemize}

CODER (Yuan et al., 2022):
\begin{itemize}
  \item Contrastive pretraining sur 87M triplets de relations UMLS
  \item Ajoute les relations (is-a, part-of, causes) en plus des synonymes dans l'objectif contrastif
  \item Medical term normalization
\end{itemize}

BioSyn (Sung et al., 2020, ACL):
\begin{itemize}
  \item Bi-encoder + cross-encoder pour entity normalization
  \item Marginal score aggregation: combine scores de plusieurs synonymes
  \item Montre que le re-ranking avec cross-encoder ameliore significativement
\end{itemize}

\subsection{Limites}

\begin{itemize}
  \item La plupart des travaux sont en anglais et sur UMLS
  \item UMLS couvre bien l'anglais mais pas le francais clinique (abreviations, synonymes informels)
  \item SapBERT et CODER ne captent pas la hierarchie complete de l'ontologie (parents, exclusions, complications)
  \item Pas d'exploitation des relations RDF/SKOS (broader, narrower, related, exactMatch)
\end{itemize}

-> on sait normaliser des mentions isolees, mais comment extraire ET normaliser sans supervision, pour des milliers de labels jamais vus a l'entrainement?


\section{Zero-shot and Open NER}
\label{sec:ie-zeroshot}

\subsection{Pourquoi zero-shot}

\begin{itemize}
  \item Les labels medicaux evoluent: nouveaux codes ICD, nouveaux criteres d'essais cliniques, nouvelles maladies (COVID-19)
  \item Impossible d'annoter des exemples pour chaque label (17K codes CIM-10)
  \item Les modeles NER classiques (BERT + linear head) ont un nombre fixe de classes de sortie
  \item -> besoin de modeles qui generalisent a des types d'entites non vus a l'entrainement
\end{itemize}

\subsection{GLiNER}

GLiNER (Zaratiana et al., 2024, NAACL):
\begin{itemize}
  \item Formulation: NER = matching entre spans de texte et descriptions de types d'entites
  \item Architecture: un seul transformer bidirectionnel encode conjointement le texte et les types d'entites
  \item Les types d'entites sont donnes en langage naturel (pas des labels fixes)
  \item Token [ENT] devant chaque type -> representation du type
  \item Span representation layer -> representation de chaque span candidat
  \item Matching par produit scalaire + sigmoid dans un espace partage
  \item Bat ChatGPT sur zero-shot NER tout en etant beaucoup plus petit (~200M params)
\end{itemize}

GLiNER2 (Zaratiana et al., 2025, EMNLP):
\begin{itemize}
  \item Extension de GLiNER a un framework multi-taches unifie: NER, relation extraction, classification, extraction structuree
  \item Meme architecture de base mais avec des tokens de relation en plus des tokens d'entite
  \item -> un seul modele pour plusieurs taches IE, toujours en zero-shot
\end{itemize}

GLiNER-BioMed (Zaratiana et al., 2025):
\begin{itemize}
  \item Adaptation de GLiNER au domaine biomedical
  \item Fine-tuning sur des datasets NER biomed (BC5CDR, NCBI, JNLPBA, etc.)
  \item +5.96\% F1 par rapport au GLiNER general sur les benchmarks biomed
  \item Montre que l'adaptation au domaine est necessaire meme pour les modeles zero-shot
\end{itemize}

-> GLiNER et ses extensions montrent que le matching span-label est une direction prometteuse. Mais ou trouver les donnees d'entrainement?

Pre-entrainement sur Pile-NER:
\begin{itemize}
  \item UniversalNER (Zhou et al., 2024, ICLR): distillation de ChatGPT
  \item ChatGPT annote des passages du Pile -> ~45K paires, 13K types d'entites distincts
  \item Instruction-tuning de LLaMA sur ces donnees
  \item Bat ChatGPT de 7-9 F1 points en moyenne sur NER
  \item GLiNER pre-entraine sur ce dataset = forte capacite zero-shot
\end{itemize}

\subsection{Autres approches}

GoLLIE (Sainz et al., 2024, ICLR):
\begin{itemize}
  \item Guideline-following Large Language model for IE
  \item Approche generative: le LLM genere des annotations structurees a partir de guidelines en langage naturel
  \item Zero-shot: les guidelines decrivent les types d'entites -> pas besoin d'exemples annotes
  \item Avantage: peut suivre des definitions complexes (criteres d'inclusion, codes CIM avec exclusions)
  \item Mais: modele genératif = plus lent et plus gros qu'un encoder
\end{itemize}

NuNER (Bogdanov et al., 2024, EMNLP):
\begin{itemize}
  \item Foundation model NER compact (RoBERTa-base, ~125M params)
  \item Pre-entraine sur annotations GPT-3.5 de C4: 24.4M mots, 4.38M annotations, 200K types d'entites
  \item Contrastive learning pour specialiser l'encoder
  \item Resultat cle: taille ET diversite du dataset d'entrainement = facteurs critiques
\end{itemize}

LLMs pour clinical NER:
\begin{itemize}
  \item Agrawal et al. (2022): LLMs pour clinical IE, performances prometteuses
  \item Singhal et al. (2022): Med-PaLM, QA medicale a l'echelle
  \item Mais: Lehman et al. (2023), ``We May Have Overestimated LLMs in Clinical NLP''
    \begin{itemize}
      \item BERT-type models restent competitifs pour les taches extractives cliniques
      \item LLMs trop gros pour deploiement hospitalier (contraintes infra)
      \item Problemes de confidentialite avec APIs proprietaires (GPT-4, Claude)
      \item -> les encoders restent pertinents pour les taches extractives cliniques
    \end{itemize}
\end{itemize}

\subsection{Limites du zero-shot actuel}

\begin{itemize}
  \item GLiNER pre-entraine sur donnees generales (Pile-NER), pas cliniques
  \item Les types d'entites sont des descriptions textuelles courtes: ne capte pas la structure hierarchique des ontologies
  \item Pas d'exploitation des relations entre codes (parent, exclusion, synonyme)
  \item Performances degradees sur les codes rares (longue traine)
\end{itemize}

-> zero-shot NER prometteur, mais: GLiNER ignore la structure ontologique, et les donnees d'entrainement sont generales. Les donnees cliniques sont privees -> comment entrainer sur du domaine clinique sans donnees reelles?


\section{Synthetic Data for Clinical NLP}
\label{sec:ie-synthetic}

\subsection{Le probleme de la rarete}

\begin{itemize}
  \item Donnees cliniques = privees (CNIL en France, HIPAA aux US)
  \item Modeles entraines sur donnees hospitaliers = non partageables entre etablissements
  \item Peu de datasets annotes publics en francais clinique: QUAERO, CAS, E3C = petits et principalement scientifiques, pas cliniques
  \item Les datasets cliniques accessibles (MIMIC) = anglais, un seul hopital americain
\end{itemize}

\subsection{LLM-as-annotator}

Paradigme recent: utiliser un LLM pour annoter des donnees, puis distiller vers un petit modele:
\begin{itemize}
  \item FineWeb-Edu (Penedo et al., 2024): Llama-3-70B annote 500K docs, distille vers classifieur BERT-like (cf Ch.2)
  \item UniversalNER (Zhou et al., 2024): ChatGPT annote le Pile pour NER, distille vers LLaMA
  \item NuNER (Bogdanov et al., 2024): GPT-3.5 annote C4 pour NER, distille vers RoBERTa
  \item Avantage: le LLM capture des connaissances medicales, le petit modele est deployable en hopital
\end{itemize}

\subsection{Generation de notes cliniques synthetiques}

Premiers travaux: generer directement des notes cliniques synthetiques avec des modeles de langue.

Ive et al. (2020, npj Digital Medicine):
\begin{itemize}
  \item Generation de notes cliniques synthetiques en psychiatrie (discharge summaries)
  \item Objectif: contourner la confidentialite des donnees
  \item Resultat: les modeles NLP entraines sur donnees synthetiques atteignent des performances comparables a ceux entraines sur donnees reelles
  \item Premier a valider experimentalement que le synthetique peut remplacer le reel pour les taches NLP cliniques
\end{itemize}

Chintagunta et al. (2021, NLPMC@ACL):
\begin{itemize}
  \item GPT-3 pour augmenter des donnees de dialogue medical: 210 exemples annotes -> performances equivalentes a 6400 exemples (~30x)
  \item Low-shot + ensemble: le LLM genere des variantes, un ensemble filtre les generations de mauvaise qualite
  \item Montre que le LLM-as-augmenter fonctionne specifiquement en domaine medical
\end{itemize}

Asclepius (Kweon et al., 2024, ACL Findings):
\begin{itemize}
  \item LLM clinique entierement entraine sur notes synthetiques
  \item Pipeline: case reports publics (PMC) -> GPT-3.5 genere des notes cliniques realistes a partir des cas
  \item Aucune donnee clinique privee utilisee
  \item Performances comparables aux modeles entraines sur MIMIC (donnees reelles)
  \item -> preuve que des notes synthetiques de qualite suffisent pour entrainer des modeles cliniques
\end{itemize}

Approche texte + annotations simultanees:
\begin{itemize}
  \item Le LLM genere le texte ET les annotations en meme temps
  \item Avantage: alignement texte-annotation garanti (pas de desynchronisation)
  \item Inputs structurees possibles: informations patient, codes CIM-10, guidelines therapeutiques
\end{itemize}

Evaluation de la qualite:
\begin{itemize}
  \item Evaluation aveugle par medecins: notes synthetiques vs reelles
  \item Criteres: coherence medicale, completude, qualite linguistique
  \item Observation recurrente: les notes synthetiques sont souvent plus completes mais moins coherentes medicalement que les notes reelles
\end{itemize}

Risques:
\begin{itemize}
  \item Style different des notes reelles (trop propre, pas assez telegraphique)
  \item Biais du LLM: peut generer des cas stereotypes
  \item Les annotations LLM ne sont pas gold-standard: heuristiques neurales, pas expertises medicales
\end{itemize}

-> les donnees synthetiques sont une solution a la rarete, mais comment evaluer correctement les modeles entraines dessus? Et plus generalement, comment comparer equitablement des modeles avec des tokenizers differents?


\section{Evaluation}
\label{sec:ie-evaluation}

\subsection{Metriques NER}

seqeval (strict entity-level):
\begin{itemize}
  \item Evaluation au niveau entite: une entite est correcte si et seulement si le type ET les frontieres sont exacts
  \item Schema IOB2: B-TYPE marque le debut, I-TYPE la continuation, O l'absence d'entite
  \item Micro-average: aggrege toutes les entites (favorise les classes frequentes)
  \item Macro-average: moyenne des scores par classe (traite toutes les classes egalement)
  \item Standard de facto pour le NER en NLP
\end{itemize}

BRATeval (CLEF eHealth 2016):
\begin{itemize}
  \item Exact match sur character offsets dans le texte original
  \item Avantage: completement independant du preprocessing et de la tokenisation
  \item Utilise comme outil officiel pour les campagnes CLEF eHealth
  \item Mais: moins repandu que seqeval dans la communaute NLP
\end{itemize}

Token-level classification (weighted F1):
\begin{itemize}
  \item Chaque token recoit un label, evaluation token par token
  \item Probleme majeur: le token ``O'' (non-entite) est largement majoritaire
  \item Le weighted F1 est domine par la classe O -> scores artificiellement eleves
  \item Utilise par DrBERT (Labrak et al., 2023): rend la comparaison avec d'autres modeles difficile
\end{itemize}

\subsection{Impact de la tokenisation}

\begin{itemize}
  \item Tokenizers differents decoupent le texte differemment
  \item SciBERT (Beltagy et al., 2019): 42\% intersection avec le vocabulaire BERT
  \item Un tokenizer qui sur-segmente les termes techniques = plus de tokens a predire = plus de chances d'erreur
  \item Consequence: les metriques token-level sont biaisees par le choix du tokenizer
  \item Exemple: ``echocardiographie'' = 3 tokens (CamemBERT) vs 2 tokens (tokenizer specialise) -> nombre de predictions different -> score different
\end{itemize}

\subsection{Besoin d'un benchmark unifie}

Problemes actuels:
\begin{itemize}
  \item Chaque papier utilise des datasets, metriques, splits, et hyperparametres differents
  \item Pas de protocole standard pour le francais biomedical
  \item CRF vs linear head, gel des couches vs fine-tuning complet, micro vs macro F1
  \item Comparaison entre modeles = difficile et souvent biaisee
\end{itemize}

Comparaison equitable entre architectures:
\begin{itemize}
  \item BERT-type (encoders) vs LLMs (decoders) vs GLiNER (bi-encoder): architectures tres differentes
  \item Tokenizers differents, procedures de fine-tuning differentes
  \item -> besoin d'une evaluation au niveau caractere, independante du tokenizer, qui permette une comparaison equitable
\end{itemize}

% TODO tableau comparatif des protocoles d'evaluation


\section{Limites et transition}
\label{sec:ie-limitations}

\begin{itemize}
  \item Rarete des donnees cliniques: les textes hospitaliers sont prives (CNIL), les modeles entraines dessus non partageables. Les corpus publics francais (QUAERO, CAS, E3C) sont petits et principalement scientifiques
    -> besoin d'alternatives aux donnees hospitalieres
  \item Label space enorme: 17K codes CIM-10, 8K CCAM, impossible d'annoter des exemples pour chaque code. Les approches supervisees classiques (BERT + linear head) ne scalent pas
    -> besoin d'approches zero-shot capables de generaliser a des labels non vus
  \item Zero-shot NER existant = pre-entraine sur donnees generales: GLiNER et UniversalNER sont entraines sur Pile-NER (texte web general), pas sur du domaine clinique. Performances degradees sur les codes rares
    -> besoin d'adapter au domaine medical
  \item Ontologies sous-exploitees dans le NER: GLiNER utilise des descriptions textuelles comme types d'entites mais ignore la structure hierarchique, les exclusions, les synonymes encodes dans RDF/SKOS
    -> besoin d'integrer la structure ontologique dans les embeddings de labels
  \item Evaluation non standardisee: pas de benchmark unifie pour le francais biomed, metriques biaisees par le tokenizer, protocoles incomparables entre papiers
    -> besoin d'un framework d'evaluation tokenizer-agnostic
  \item Encoders vs decoders: malgre l'engouement pour les LLMs, les encoders restent competitifs pour les taches extractives cliniques (Lehman et al., 2023) et sont deployables sur les infrastructures hospitalieres
\end{itemize}

-> ce chapitre a couvert les taches de clinical IE: NER, codage medical, normalisation, et les defis specifiques du domaine clinique. Les limites identifiees (rarete, label space, zero-shot non adapte, ontologies sous-exploitees, evaluation non standardisee) definissent les axes de recherche de la partie suivante.

% TODO rediger la transition proprement
